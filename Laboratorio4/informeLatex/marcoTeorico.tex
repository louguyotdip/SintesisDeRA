
%\begin{document}
 \section{Introducción teórica}
  \vspace{1em}
  
   Un filtro activo es un filtro analógico distinguido por el uso de uno o más componentes 
  activos diferenciándose de los filtros pasivos los cuales solamente usan componentes 
  pasivos. Típicamente, este elemento activo puede ser un transistor o un amplificador 
  operacional. 
  
  Un filtro activo puede presentar ganancia en toda o parte de la señal de salida respecto 
  a la señal de entrada. En su implementación se combinan elementos activos y pasivos, 
  siendo frecuente el uso de amplificadores operacionales, que permite obtener resonancia 
  y un elevado factor Q sin el empleo de bobinas. 
  Los filtros activos se realizan en base a una planilla de requerimientos solicitados, que 
  nos proporcionan las características que tendrá el mismo. Para hallar la función de 
  transferencia apropiada para estos requisitos, se aplica la teoría de la Aproximación, que 
  entre ellas podemos encontrar: 
  \begin{itemize}
  	\item Butterworth
  	\item Chebyschev
  	\item Cahuer
  	\item etc.		
  \end{itemize}
 
   Mediante estas aproximaciones se obtiene entonces la función de transferencia de 
  adecuada. Esta función final se la puede expresar como una productoria de 
  bicuadráticas, lo que se muestra en la siguiente ecuación: 
  \vspace{1em}
  
  \begin{figure}[h!]
  	\centering
  	\includegraphics[width=0.6\linewidth]{ecuacionFiltros.png}
  	\label{fig:enter-label}
  \end{figure}
  
   Con el motivo de modelar individualmente distintas etapas con diversas topologías 
  obteniendo así los componentes físicos que corresponden a cada una de estas. Luego 
  mediante una conexión en cascada se obtiene el filtro esperado. 
  En lo que respecta a las topologías aplicadas para la síntesis de las funciones, podemos 
  destacar tres de ellas:
  \begin{itemize}
  	\item Topología de Realimentación positiva
  	\item Topología de Realimentación negativa
  	\item Topología de 4 operacionales	
  \end{itemize}
   Utilizando algunas de ellas se llevará a cabo el presente trabajo con el fin de modelar un 
  filtro pasabanda. 
  

%\end{document}